\documentclass{exam}
\usepackage{../../scripts/meta}

%%% Change these %%%%%%%%%%%%%%%%%%%%%%%%%%%%%%%%%%%%%%%%%%%%%%%%%%%%%%%%%%%%%%
\discnumber{8}
\title{}
\date{Distributions, Variance, Inequalities, Confidence Intervals}

%%%%%%%%%%%%%%%%%%%%%%%%%%%%%%%%%%%%%%%%%%%%%%%%%%%%%%%%%%%%%%%%%%%%%%%%%%%%%%%
\begin{document}
\maketitle
\rule{\textwidth}{0.15em}
\fontsize{12}{15}\selectfont
\thispagestyle{empty}


%%% General Notes %%%%%%%%%%%%%%%%%%%%%%%%%%%%%%%%%%%%%%%%%%%%%%%%%%%%%%%
\section{General Comments}
\begin{questions}
\item Important Questions
\begin{itemize}
\item 1.1 - 1
\item 1.1 - 3,5 are esp. Good
\item \#5 on pg. 9
\item 4 in continuous distributions
\end{itemize}

\item They should be decently familiar with Markov/Chebyshev’s Inequalities – so make sure to focus on those, and that they are comfortable with them moving on. They are tested quite often
\item I think it’s good to present confidence intervals as an application of the inequalities, which should make it more intuitive. Up to you, whatever makes most sense to you and your students
\item Conditional Expectation is difficult. This will be covered in lecture Friday, so students will not have seen it, or be very confused about it. Make sure to prepare an explanation
\item Continuous has the lowest priority due to it being covered (most likely) after Thanksgiving
\item Make sure they understand why we integrate/derive when to manipulate between CDF/PDF
\item Just like the intro to RV’s recently, have them be comfortable with defining continuous RV’s
\item Move onto continuous distributions only if you have time. I think it’s intuitive to present them as the continuous analogs to their discrete counterparts – but again, do whatever works best
\item Make sure they understand why we use the values that we use for PDF’s/CDF’s
A lot of the problems here are just plug and chug, so shouldn’t be too difficult
\end{questions}

\section{Questions}
\subsection{Conditional Expectation}
\begin{enumerate}
\subimport{../../topics/conditional_expectation/meta/}{intro.tex}
\end{enumerate}
\end{document}