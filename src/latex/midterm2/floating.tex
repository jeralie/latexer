\question \textbf{Floating Outside of Reality}\newline
Assume we have a computer with a finite number of infinite-precision floating point numbers, show, given a fixed starting state of the computer, that there exist real numbers we cannot calculate. How many such numbers are there?
\begin{solution}
Remember that the set of all programs is countable, since, in order to write them, they must be formed from a finite length sequence in some alphabet. Since the starting state of the computer is fixed, and each program maps the starting state to the final state of the computer, there is a countably infinite number of states we can reach. In order to have every real number be computable, we would have to have an uncountable infinity of states. Therefore it is impossible to calculate every real number.\newline
There must be an uncountable infinity of these numbers, since, as we have already reasoned, there is a countable infinity of calculable reals, so the set difference between the reals and the calculable reals must be uncountable (the union of the countable calculable reals and the incalculable reals must be uncountable, the union of two countable sets is countable).

\end{solution}