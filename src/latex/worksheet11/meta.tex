\documentclass{exam}
\usepackage{../../scripts/meta}

%%% Change these %%%%%%%%%%%%%%%%%%%%%%%%%%%%%%%%%%%%%%%%%%%%%%%%%%%%%%%%%%%%%%
\discnumber{11}
\title{}
\date{Markov Chains, Continuous Probability, Conditional Expectation}

%%%%%%%%%%%%%%%%%%%%%%%%%%%%%%%%%%%%%%%%%%%%%%%%%%%%%%%%%%%%%%%%%%%%%%%%%%%%%%%
\begin{document}
\maketitle
\rule{\textwidth}{0.15em}
\fontsize{12}{15}\selectfont
\thispagestyle{empty}


%%% General Notes %%%%%%%%%%%%%%%%%%%%%%%%%%%%%%%%%%%%%%%%%%%%%%%%%%%%%%%
\section{General Comments}
\begin{itemize}
\item Definitely allocate time for conditional expectation if you didn't get a chance to do it last week, especially the proofs
\item Gauge linear algebra skills before going too far in-depth. They should know how to multiply a matrix by a vector, etc. before starting
\item Go over definitions first before going into problems – make sure they understand what irreducible, periodic, etc. mean. Draw out basic examples first. Understanding how to use the first-step equations to find everything else is especially important

\item Make sure they understand how a balance equation works. As in, why we would multiply the transition matrix by a distribution to get the distribution at the next time step.




\item Life of Alex is a really good question to go all the way through to ensure that students understand the basics of Markov Chains. Bet On It is also very good

\item This is one of our shorter worksheets, so try to get to at least touch on every problem, but don't worry about it if you can. It's better to completely finish some of the problems than only do part of all the problems
\end{itemize}


\section{Questions}

\end{document}

