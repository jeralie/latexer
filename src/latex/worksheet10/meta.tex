Week 11: LLSE, Covariance, Conditional Expectation, and Markov Chains

Focus on
LLSE: all parts
Covariance: 2.1 and 2.2.1
Conditional Expectation: all parts

Covariance
General: Make the connection between the covariance formula and the variance formula, and why it makes sense to define covariance this way.
2.1: Plug and chug; be explicit about when independence of the random variables is used.
2.2:
	1: This goes back to basic expectation and probability; students who can do 2.1 but get stuck here likely have some gaps in their understanding of previous topics.
	2: Back to basics; given the actual distribution, calculate expectation (and thus covariance) from scratch.
Typo in solutions: Covariance section 1.2 number 1 the line E(AB) = E(AB) - E(A)E(B) should be Cov(A, B) = E(AB)-E(A)E(B) which you can show is true from the definition of covariance, which is that Cov(A,B) = E((A-E(A))(B-E(B))) 
= E(AB - AE(B) - BE(A) + E(A)E(B)) 
= E(AB) - E(A)E(B) - E(B)E(A) + E(A)E(B) 
= E(AB) - E(A)E(B)

LLSE
Make sure to do covariance and conditional expectation first because LLSE relies on it
General: Explain what LLSE is, and why the formula makes sense--intuitively, why. If there is time or interest, prove it.
1.1: Plug and chug; be explicit about when independence of the random variables is used.


Conditional Expectation
General: Make the connection between normal expectation and conditional expectation, and explain intuitively what conditional expectation means
3.1: 
	1-2: Using many properties of summations, make sure students understand each step
	3. Plug and chug, going back to basic distributions.

Markov Chains
General: Explain the idea of a Markov chain, specifically the first order Markov property, and homogeneity:
	First order Markov property: probability of an object being in any state depends only on the state the object was in right before.
	Homogeneity: Regardless of how many steps the chain has run, the transition matrix doesn’t change (i.e. P(X_{i+1} = a | X_{i} = b) is the same regardless of the value of i)
	Irreducibility: there is no sink
	A state x is periodic if by starting in x, all ways of returning to x have some GCD greater than 1. If any state is periodic, then the chain is periodic.
	Explain the intuition behind stationary/invariant distributions and irreducibility and aperiodicity
4.1: Basic uses of the formulas.
	d: Do not forget that any distribution must add up to 1, and explain that using this method of finding the stationary distribution, two of the first-step equations will be linearly dependent.

