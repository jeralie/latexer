\documentclass{exam}
\usepackage{../../scripts/meta}

%%% Change these %%%%%%%%%%%%%%%%%%%%%%%%%%%%%%%%%%%%%%%%%%%%%%%%%%%%%%%%%%%%%%
\discnumber{8}
\title{}
\date{Distributions, Variance, Inequalities, Confidence Intervals}

%%%%%%%%%%%%%%%%%%%%%%%%%%%%%%%%%%%%%%%%%%%%%%%%%%%%%%%%%%%%%%%%%%%%%%%%%%%%%%%
\begin{document}
\maketitle
\rule{\textwidth}{0.15em}
\fontsize{12}{15}\selectfont
\thispagestyle{empty}


%%% General Notes %%%%%%%%%%%%%%%%%%%%%%%%%%%%%%%%%%%%%%%%%%%%%%%%%%%%%%%
\section{General Comments}
\begin{questions}
\item In distributions section, ok to just do expectation and variance in terms of trials (instead of days/minutes or whatever, unless you want to make the point that variance’s units are in units squared)
\item 1.1:1) c) is probably skippable (definitely depends on the variance section later, if your do this problem you should go over variance section first)
\item Important questions to get to:
\begin{itemize}
\item If your students are struggling on Expectation and Variance, 2.1
\item If your students are feeling fine with Expectation/Variance calculations, skip straight to 2.2/2.3
\item 2.4/2.5 is necessary for everyone
\item 3.1
\item 3.3
\item 3.4
\item 3.5
\end{itemize}

\item Make sure your students can calculate Expectation and Variance; they should be pretty  comfortable with it considering it was on last week's worksheet
\item Distributions should also be easy; go over the expectations/variances of the important ones
\item The proofs with Variance are important; variance proofs appear a lot
\item For most people, these bounds will be new (taught Monday or Wednesday), but they’re not terrible difficult to teach. Prove Markovs if they are uncomfortable with it
\item Only Friday sections will probably get to CIs, will be repeated next week most likely. 

\item Most of this stuff is pretty easy to teach; Distributions, Expectation, and Variance are pretty basic to teach; just make sure you are getting your math right (review the worksheet); make sure to explain how to find E(f(X)) in order to find E(X2)
\item 2.6/2.7 will require a lot of steps; review each one carefully before teaching these
The best way to talk about inequalities is to derive them all
\end{questions}

\section{Questions}
 
\end{document}
