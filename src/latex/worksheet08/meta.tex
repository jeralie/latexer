\documentclass{exam}
\usepackage{../../scripts/meta}

%%% Change these %%%%%%%%%%%%%%%%%%%%%%%%%%%%%%%%%%%%%%%%%%%%%%%%%%%%%%%%%%%%%%
\discnumber{8}
\title{}
\date{Distributions, Variance, Inequalities, Confidence Intervals}

%%%%%%%%%%%%%%%%%%%%%%%%%%%%%%%%%%%%%%%%%%%%%%%%%%%%%%%%%%%%%%%%%%%%%%%%%%%%%%%
\begin{document}
\maketitle
\rule{\textwidth}{0.15em}
\fontsize{12}{15}\selectfont
\thispagestyle{empty}

\section{General Comments}
\begin{questions}
\item In case you didn't have time to cover distributions last week, make sure to do them, building an intuition for why the expectations make sense. Intuitions for variance aren't as important, but make sure they understand the formulas (no need to memorize, really).

\item The questions in Variance sections are really good, so cover them as much as you can so they can get practice. Red solo cup one and independent RV ones are especially good ones.

\item Draw diagrams and explain intuition for Markov/Chebyshev (balance scale example often brought up in lecture). Make sure you cover how to use Markov to prove Chebyshev question. Make sure they are aware of the restrictions on the random variables that Markov requires. Recommand to draw diagrams to help students to understand what Chebyshev bound is calculating.

\item Don't worry if you don't have time to get to confidence intervals, as we'll probably cover it next week too. Prove the formulas behind CIs and make sure that they can relate it to other inequalities they've seen.


%%% General Notes %%%%%%%%%%%%%%%%%%%%%%%%%%%%%%%%%%%%%%%%%%%%%%%%%%%%%%%


% \item Make sure your students can calculate Expectation and Variance; they should feel comfortable calculating it for different Random Variables before moving on.
% \item Distributions should also be easy; go over the expectations/variances of the important ones
% \item The proofs with Variance are important; variance proofs appear a lot
% \item For most people, these bounds will be new (taught Monday or Wednesday), but they’re not terribly difficult to teach. Prove Markovs if they are uncomfortable with it. Make sure they are aware of the restrictions on the random variables that Markov requires. Recommand to draw diagrams to help students to understand what Chebyshev bound is calculating.

% \item Only Friday sections will probably get to CIs, will be repeated next week most likely. 

% \item Most of this stuff is pretty easy to teach; Distributions, Expectation, and Variance are pretty basic to teach; just make sure you are getting your math right (review the worksheet); make sure to explain how to find E(f(X)) in order to find E(X2)
% \item 2.6/2.7 will require a lot of steps; review each one carefully before teaching these
% The best way to talk about inequalities is to derive them all
\end{questions}

\section{Questions}
 
\end{document}
