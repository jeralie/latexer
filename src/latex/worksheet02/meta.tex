\documentclass{exam}
\usepackage{../../scripts/meta}

%%% Change these %%%%%%%%%%%%%%%%%%%%%%%%%%%%%%%%%%%%%%%%%%%%%%%%%%%%%%%%%%%%%%
\discnumber{0}
\title{}
\date{Graphs, Trees, Hypercubes}

%%%%%%%%%%%%%%%%%%%%%%%%%%%%%%%%%%%%%%%%%%%%%%%%%%%%%%%%%%%%%%%%%%%%%%%%%%%%%%%
\begin{document}
\maketitle
\rule{\textwidth}{0.15em}
\fontsize{12}{15}\selectfont
\thispagestyle{empty}


%%% Include topics here %%%%%%%%%%%%%%%%%%%%%%%%%%%%%%%%%%%%%%%%%%%%%%%%%%%%%%%
\section{General Comments - Graph Theory}
\begin{enumerate}
\item Make sure to take attendance!
\item Emphasize that the worksheet this week is extremely long, it is not meant to be completely covered in 1.5 hours -- they should practice the problems that you didn’t get to
\item Make sure that you are comfortable with the subtle differences between a walk and path or a tour and cycle. 
\item Especially emphasize 3-cycle and/or ham-path in Graph Theory.
\item If you are teaching in the beginning of the week, this will most likely be their first exposure to proving things about graphs. Go slow and make sure they understand why each step is necessary and how you would think of it
\item If low on time, skip Eulerian tours
\item Have students take special note of the 4 properties of trees we list. Some good general tree advice is to have those 4 written out exactly, which makes for a much easier time coming up with proofs
\item For Rooted Tree, a visual/intuitive explanation rather than a formal proof is fine. 
\item If there is time, try and emphasize the Spanning Tree problem.
\item Hypercubes; M/T/W: just talk them through the definition and ask if they need anything clarified from lecture. Take a look at the lecture notes before teaching to see what they covered
\item Remember that there are two major definitions of a hypercube. 
\item TH/F: they will probably be more comfortable with hypercubes. Do these 2 questions as time permits.
\item The 2 questions are nice since you can use the recursive definition to make one pretty easy and the binary definition to do the other one, driving home this point of both representations being imortant. You can also use a visual definition if you feel that it can convey the intuition better.
\end{enumerate}

\section{General Comments - Modular Arithmetic}
\begin{enumerate}
\item FLT Proof is important to go through; make sure they understand every step of the way. It's also not necessary to go through both. Do so if you have time, but skip last one if low on time
\item CRT Proof is not necessary to go through, only if they really have trouble understanding it (proof is rather lengthy). The eggs question is pretty much just an application of the first question, which just uses numbers. Probably not necessary to go through both. Make sure that they understand how to go through these problems mechanically
\end{enumerate}

\section{General Comments - RSA}
\begin{enumerate}
\item Give a mini-lecture about RSA; Try and draw a picture to explain the different components of RSA to your students.
\item Definitely try and go over #3, as it provides a lot of intuition on RSA.
\item If time, try and do #1, but if not, encourage students to work on #1 on their own time as it is a purely algebraic problem.
\item Stress the inability to factor integers as one of the reasons why RSA works.
\end{enumerate}


\clearpage 

%%% Include topics here %%%%%%%%%%%%%%%%%%%%%%%%%%%%%%%%%%%%%%%%%%%%%%%%%%%%%%%
\section{Questions}
\subsection{Graph Theory}
\begin{enumerate}
\subimport{../../topics/graphs/meta/}{3_cycle.tex}
\subimport{../../topics/graphs/meta/}{build_up_error.tex}
\end{enumerate}

\subsection{Hypercubes}
\begin{enumerate}
\subimport{../../topics/hypercubes/meta/}{introduction.tex}
\end{enumerate}
%%%%%%%%%%%%%%%%%%%%%%%%%%%%%%%%%%%%%%%%%%%%%%%%%%%%%%%%%%%%%%%%%%%%%%%%%%%%%%%


\end{document}
