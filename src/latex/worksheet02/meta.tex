\documentclass{exam}
\usepackage{../../scripts/meta}

%%% Change these %%%%%%%%%%%%%%%%%%%%%%%%%%%%%%%%%%%%%%%%%%%%%%%%%%%%%%%%%%%%%%
\discnumber{0}
\title{}
\date{Graphs, Trees, Hypercubes}

%%%%%%%%%%%%%%%%%%%%%%%%%%%%%%%%%%%%%%%%%%%%%%%%%%%%%%%%%%%%%%%%%%%%%%%%%%%%%%%
\begin{document}
\maketitle
\rule{\textwidth}{0.15em}
\fontsize{12}{15}\selectfont
\thispagestyle{empty}


%%% Include topics here %%%%%%%%%%%%%%%%%%%%%%%%%%%%%%%%%%%%%%%%%%%%%%%%%%%%%%%
\section{General Comments}
\begin{enumerate}
\item Make sure to take attendance!
\item Emphasize that the worksheet this week is suuuuuuuuperr long, it is not meant to be completely covered in 1.5 hours -- they should practice the problems that you didn’t get to.
\item Make sure that you are comfortable with the subtle differences between a walk and path or a tour and cycle. 
\item Especially emphasize \#4, \#5, \#6 in Graph Theory
\item If you are teaching in the beginning of the week, this will most likely be their first exposure to proving things about graphs. Go slow and make sure they understand why each step is necessary and how you would think of it
\item If low on time, skip Eulerian tours
\item Have students take special note of the 4 properties of trees we list. Some good general tree advice is to have those 4 written out exactly, which makes for a much easier time coming up with proofs
\item Especially emphasize \#1, \#5, \#6 
\item \#2, \#3 could potentially get removed. If they do not then replace \#5 wtih \#2 on the required questions list
\item Skip \#7 if short on time
\item You may want to take some time to prove the theorem
\item Hypercubes; M/T/W: just talk them through the definition and ask if they need anything clarified from lecture. Take a look at the lecture notes before teaching to see what they covered
\item Remember that there are two major definitions of a hypercube. 
\item TH/F: they will probably be more comfortable with hypercubes. Do these 2 questions as time permits.
\item The 2 questions are nice since you can use the recursive definition to make one pretty easy and the binary definition to do the other one, driving home this point of both representations being imortant.
\end{enumerate}

\clearpage 

%%% Include topics here %%%%%%%%%%%%%%%%%%%%%%%%%%%%%%%%%%%%%%%%%%%%%%%%%%%%%%%
\section{Questions}
\subsection{Graph Theory}
\begin{enumerate}
\subimport{../../topics/graphs/meta/}{3_cycle.tex}
\subimport{../../topics/graphs/meta/}{build_up_error.tex}
\end{enumerate}

\subsection{Hypercubes}
\begin{enumerate}
\subimport{../../topics/hypercubes/meta/}{introduction.tex}
\end{enumerate}

\subsection{Bijections}

%%%%%%%%%%%%%%%%%%%%%%%%%%%%%%%%%%%%%%%%%%%%%%%%%%%%%%%%%%%%%%%%%%%%%%%%%%%%%%%


\end{document}
