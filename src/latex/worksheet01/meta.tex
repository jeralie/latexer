\documentclass{exam}
\usepackage{../../scripts/meta}

%%% Change these %%%%%%%%%%%%%%%%%%%%%%%%%%%%%%%%%%%%%%%%%%%%%%%%%%%%%%%%%%%%%%
\discnumber{1}
\title{}
\date{Quantifiers, Methods of Proof}

%%%%%%%%%%%%%%%%%%%%%%%%%%%%%%%%%%%%%%%%%%%%%%%%%%%%%%%%%%%%%%%%%%%%%%%%%%%%%%%
\begin{document}
\maketitle
\rule{\textwidth}{0.15em}
\fontsize{12}{15}\selectfont
\thispagestyle{empty}


%%% General Notes %%%%%%%%%%%%%%%%%%%%%%%%%%%%%%%%%%%%%%%%%%%%%%%%%%%%%%%
\section{General Comments}
\begin{questions}
\item Logistics: 
\begin{enumerate}[label=(\alph*)]
\item Mentors who have section Monday will not have section the first week. They can either try to reschedule their section for later in the week, or their students can sign up for a different section on the spreadsheet on Scheduler
\item Book rooms!
\item Don’t forget to take attendance during section!
\item Tell your students to check Piazza for worksheets! (I’ll add you soon)
\end{enumerate}
\item Start off class with checking to see if they have any questions but don’t spend more than about 20 minutes on this, but do make sure they get induction!
\item If students have any questions about Worksheet 0 (so make sure and review this first!)
\begin{enumerate}[label=(\alph*)]
\item Work on Fibonnacci question iff they were very comfortable with induction from worksheet 0 - if not, spend some more time on induction because it’s very important. 
\end{enumerate}
\item A lot of the stable marriage is filling in proofs: For this reason, it’s important to lecture on this stuff specifically. A lot of the students don’t read the notes and this is straight from the notes
\end{questions}

\clearpage 

%%% Include topics here %%%%%%%%%%%%%%%%%%%%%%%%%%%%%%%%%%%%%%%%%%%%%%%%%%%%%%%
\section{Questions}
\subsection{Stable Marriage}
\begin{enumerate}
\subimport{../../topics/stable_marriage/meta/}{please.tex}
\end{enumerate}

\subsection{Well Ordering Principle}
\begin{enumerate}
\subimport{../../topics/wop/meta/}{infimum.tex}
\end{enumerate}

%%%%%%%%%%%%%%%%%%%%%%%%%%%%%%%%%%%%%%%%%%%%%%%%%%%%%%%%%%%%%%%%%%%%%%%%%%%%%%%

\end{document}
