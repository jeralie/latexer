\documentclass{exam}
\usepackage{../../scripts/meta}

%%% Change these %%%%%%%%%%%%%%%%%%%%%%%%%%%%%%%%%%%%%%%%%%%%%%%%%%%%%%%%%%%%%%
\discnumber{3}
\title{CRT, Bijections, FLT, RSA}
\date{September 18 to September 22, 2017}

%%%%%%%%%%%%%%%%%%%%%%%%%%%%%%%%%%%%%%%%%%%%%%%%%%%%%%%%%%%%%%%%%%%%%%%%%%%%%%%
\begin{document}
\maketitle
\rule{\textwidth}{0.15em}
\fontsize{12}{15}\selectfont
\thispagestyle{empty}


%%% General Notes %%%%%%%%%%%%%%%%%%%%%%%%%%%%%%%%%%%%%%%%%%%%%%%%%%%%%%%
\section{General Comments}
\begin{questions}
\item Logistics
\begin{itemize}
\item There will be no sections on Monday; the midterm is then. 
\end{itemize}
\item Bijections
\begin{itemize}
\item Make sure that they understand one-to-one = injective and onto = surjective
\item Make sure to go over what it means to be a well-defined function, this is \textit{essential} to understanding injective/surjective.
\item Don't be afraid to spend a good chunk of time talking through the drawings.
\item Many students will assume that surjections are just injections from the codomain to the domain (they are not!).
% \item[injection] For a function $f : C \to D$ $\forall x, y \in C\ldotp f(x) = f(y) \implies x = y$
% \item[surjection] For a function $f : C \to D$ $\forall y \in D\ldotp \exists x \in C\ldotp f(x) = y$
\item The “why is this mapping (not) bijective” questions are important
\item Make sure they’re comfortable with bijections (this will be especially important in RSA)
\item Emphasize that bijection is equivalent to invertibility.
\end{itemize}
\item FLT
\begin{itemize}
\item Last question is kind of repetitive – skip if low on time
\end{itemize}

\item RSA
\begin{itemize}
\item Sections earlier in the week may not have strong RSA practice, so don’t spend too much time if they aren’t very familiar with it
\item If you don’t get to the RSA questions, briefly explain how it works on a high-level
\item Make sure they understand how RSA actually works – the implementation questions test for that pretty well
\item Draw a picture! Ask what is public? What is private?
\item Coin tosses question is interesting. Tests if they actually understand why RSA works, rather than just how it’s implemented
\item Go over the proof from notes on how/why RSA works
\end{itemize}

\item CRT
\begin{itemize}
\item This may or not be covered. 
% UPDATE THIS^
\item This is something that students have a bit of trouble with, but I'd say work through a simple case. 
\end{itemize}

\item Polynomials
\begin{itemize}
\item Monday-Wednesday will probably not get to this
\end{itemize}

\item Secret Sharing
\begin{itemize}
\item This is in there just for Friday people
\end{itemize}
\item Mandatory questions you have to get to
\begin{itemize}
\item Bijections:  Why you can’t find injections/bijections between some spaces
\item FLT: FLT Proof
\item RSA: That one proof about how it works and applies FLT
\item Polynomials: Only do this if you have time, if you get to this do a vanilla intro
\item Secret Sharing: Again do a vanilla intro if you get to this section
\end{itemize}

\end{questions}
\end{document}
