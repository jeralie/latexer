\question \textbf{Bound It}\newline
A random variable $X$ is always strictly larger than -100. You know 
that $\E(X) = −60$. Give the best upper bound you can on $\P[X \geq −20]$.
    %\vspace{2.5cm}
\begin{solution}[3cm]
Notice that we do not have the variance of $X$, so Chebyshev's bound 
is not applicable here. There is no upper bound on $X$, so Hoeffding’s 
inequality cannot be used. We know nothing else about its distribution 
so we cannot evaluate $\E[esX]$ and so Chernoff bounds are not available. 
Since $X$ is also not a sum of other random variables, other bounds or 
approximations are not available. This leaves us with just Markov's 
Inequality. But Markov Bound only applies on a non-negative random 
variable, whereas $X$ can take on negative values.
					
This suggests that we want to “shift” $X$ somehow, so that we can apply 
Markov’s Inequality on it. Define a random variable $Y = X + 100$, 
which means $Y$ is strictly larger than 0, since $X$ is always strictly 
larger than −100. Then, $\E(Y) = \E(X +100) = \E(X)+100 = −60+100 = 40$. 
Finally, the upper bound on $X$ that we want can be calculated via $Y$, 
and we can now apply Markov's Inequality on $Y$ since $Y$ is strictly 
positive.
					
$\P[X \geq −20] = \P[Y \geq 80] \leq \frac{\E(Y)}{80} = \frac{40}{80} = 
\frac{1}{2}$
					
Hence, the best upper bound on $\P[X \geq −20]$ is $\frac{1}{2}$. 
\end{solution}