{\tabulinesep=1mm
\begin{tabu}{|p{16cm} |}
\hline
\vspace{2 mm}
\textbf{Theorem 1} : Distributing $k$ distinguishable balls into $n$ 
distinguishable boxes, with exclusion, corresponds to forming a 
permutation of size $k$, taken from a set of size $n$. Therefore, 
there are $P(n, k)  = n_k  = n*(n - 1)*(n - 2) \dotsc (n - k+1)$ 
different ways to distribute $k$ distinguishable balls into $n$ 
distinguishable boxes, with exclusion  \vspace{5 mm}

\textbf{Theorem 2} : Distributing $k$ distinguishable balls into $n$ 
distinguishable boxes, without exclusion, corresponds to forming a 
permutation of size $k$, with unrestricted repetitions, taken from a 
set of size $n$. Therefore, there are $n^k$ different ways to 
distribute $k$ distinguishable balls into $n$ distinguishable boxes, 
without exclusion.\vspace{5 mm}

\textbf{Theorem 3} : Distributing $k$ indistinguishable balls into 
$n$ distinguishable boxes, with exclusion corresponds to forming a 
combination of size $k$, taken from a set of size $n$. Therefore, 
there are $C(n, k) =  { n \choose k}$ different ways to distribute 
$k$ indistinguishable balls into $n$ distinguishable boxes, with 
exclusion. \vspace{5 mm}
\\
\hline
\end{tabu}
}