\question How many solutions does $x + y + z = 10$ have, if all variables 
must be positive integers? 

\begin{solution}[1 in]
We know no number can be greater than 8, because all are positive.
So position: $y$ can take on any value from $1 \rightarrow 8$, and $z$ will just be 
whatever is left ($y$ can only each of 8 values of $x$ 
($1 \rightarrow 8$), we solve $y + z = 10 - x$
Take example $x = 1$, then $y + z = 9$. Because each value $\ge 1$, there 
are 8 solutions for this equality take on 8 values because $z \ge 1$).
So we can see that in general, there are $10-x-1$ solutions for each 
value of $x$.
so when $x = 1$, 8 solutions; when $x = 2$, 7 solutions, etc. for a total 
of $8 + 7 + 6 + 5 + 4 + 3 + 2 + 1$ (1 happens when $x=8$ and $y$ and $z$ both must $= 1$)
total $= 8 + 7 + 6 + 5 + 4 + 3 + 2 + 1 = 36 $ solutions

It's easier to think in terms of stars and bars. Bars can't be next to each 
other since variables are all positive integers, and this would imply that one of the values is 0. So $n = 10$ stars, 
$k = 3$ bars. Answer = ${n-1 \choose k-1} = {9 \choose 2} = 36$
\end{solution}