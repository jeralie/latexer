\question Imagine that in the context of stable marriage all 
men have the same preference list. That is to say there is a 
global ranking of women, and men's preferences are directly 
determined by that ranking.
Use any method of proof to answer the following questions.

\begin{questions}
\item Prove that the first woman in the ranking has to be paired with 
her first choice in any stable pairing.
\begin{solution} [1.2 in]
 If the first woman is not paired with her first choice, then she and 
 her first choice would form a rogue couple, because her first choice 
 prefers her over any other woman, and vice versa.
\end{solution}

\item Prove that the second woman has to be paired with her first 
choice if that choice is not the same as the first woman's first choice. 
Otherwise she has to be paired with her second choice.
\begin{solution}[1.2 in]
If the first and second women have different first choices, then the 
second woman must be matched to her first choice. Otherwise she and her 
first choice would form a rogue couple (since her first choice is not 
matched to the first woman, he would prefer the second woman over his 
current match).
If the first choices are the same, then the second woman must be paired 
with her second choice, otherwise she and her second choice would form 
a rogue couple (neither of them are matched to their first choices, 
and they are each other's second choice).
\end{solution}

\item Continuing this way, assume that we have determined the pairs 
for the first $k-1$ women in the ranking. Who should the $k$-th woman 
be paired with?
\begin{solution}[1.2 in]
The $k$-th woman should be paired with the first man on her list who 
has not been matched yet (with the first $k-1$ women). If she's not 
matched to him, they would form a rogue couple. This is because the 
man would have to be matched to a woman ranked worse than $k$, so she 
would prefer the $k$-th woman over his current partner, and the $k$-th 
woman obviously prefers him to whoever she's matched with.
\end{solution}

\item  Prove that there is a unique stable pairing.
\begin{solution}
In the previous parts, we saw that for each woman, given the pairs 
for the lower-ranked women, her pair would be determined uniquely. So 
there is only one stable pairing.
This can be stated and proved more rigorously using induction. Namely 
that there is a unique pairing for the first $k$ women, assuming 
stability. An induction on $k$ would prove this.
\end{solution}
\end{questions}