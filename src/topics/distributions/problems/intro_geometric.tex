{\tabulinesep=1mm
\begin{tabu}{|p{16cm} |}
\hline
\vspace{1 mm}
\textbf{Geometric Distribution: Geom(p)}
Number of trials required to obtain the first success. Each trial has 
probability of success equal to $p$. The probability of the first success 
happening at trial $k$ is:
\[ \P[X = k] = (1 - p)^{k - 1}*p, k > 0\]
The expectation of a geometric distribution is:
\[\E(X) = \frac{1}{p}\]
\vspace{1 mm}
\\
\hline
\end{tabu}
}

\begin{solution} % walkthrough of deriving E(X) for Geometric
Derivation of $\E(X)$:
The clever way to find the expectation of the geometric distribution 
uses a method known as the renewal method. $\E(X)$ is the expected number 
of trials until the first success. Suppose we carry out the first trial,
and one of two outcomes occurs. With probability p, we obtain a success 
and we are done (it only took 1 trial until success). With probability 
$1 - p$, we obtain a failure, and we are right back where we started. 
In the latter case, how many trials do we expect until our first success? 
The answer is $1 + E(X)$: we have already used one trial, and we expect 
$\E(X)$ more since nothing has changed from our original situation 
(the geometric distribution is memoryless). Hence 
$\E(X) = p * 1 + (1 - p) * (1 + E(X))$
\end{solution}