\question In RSA, if Alice wants to send a confidential message to Bob, 
she uses Bob’s public key to encode it. Then Bob uses his private key 
to decode the message. Suppose that Bob chose $N = 77$. And then Bob 
chose $e = 3$ so his public key is (3, 77). And then Bob chose $d = 26$ 
so his private key is (26, 77). \newline
Will this work for encoding and decoding messages? If not, where did 
Bob first go wrong in the above sequence of steps and what is the 
consequence of that error? If it does work, then show that it works.

\begin{solution}[2 in]
$e$ should be co-prime to $(p - 1)(q - 1)$.\newline
$e = 3$ is not co-prime to $(7 - 1)(11 - 1) = 60$, so this is incorrect, 
since therefore $e$ does not have an inverse $\mod 60$. 
\end{solution}