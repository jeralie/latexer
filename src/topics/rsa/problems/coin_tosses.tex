\question \textbf{Coin tosses over text messages} \newline
You and one of your friends want to get your hands on the new gadget 
that’s coming out. One of you has to wait in line overnight so that 
you have a chance to get the gadgets while they last. In order to 
decide who this person should be, you both agree to toss a coin. But 
you won’t meet each other until the day of the actual sale and you 
have to settle this coin toss over text messages (using your old gadgets). 
Obviously neither of you trusts the other person to simply do the coin 
toss and report the results.\newline
					
How can you use RSA to help fix the problem?	
\begin{solution}
If there was a way for me to make my choice (i.e. toss the coin) 
without revealing to my friend what the result was before s/he makes 
her/his decision, then we would be in good shape.
RSA enables us to do just that. One can commit to a choice without 
revealing what that choice really is. So here is how we proceed: 
\begin{enumerate}

\item First I select a public key (N,e) and a private key d. I toss a 
coin, but instead of sending the result to my friend, I first encrypt 
it using the public key (N,e). Then I send my friend the public key 
along with the encrypted message. 
\item My friend is supposedly (read the next part for why the word 
supposedly is used) unable to see what the result of the coin toss was 
and therefore cannot cheat. So s/he makes her/his choice (what HEADS 
and TAILS mean) and sends it to me.	
\item Once I have successfully received the result, I reveal the 
result of the coin toss by sending my friend the result in plain text 
(i.e. with no encryption). My friend can now verify that I have not 
cheated (i.e. I have not changed the result) by encrypting the result 
using the public key I have given her/him and making sure it was the 
same as the encrypted message I send her/him. Note that RSA encryption 
and decryption are both bijections, therefore if I know the encrypted 
version of two messages are the same, then those two messages must be 
the same.	
\end{enumerate}					
Note that I cannot cheat here, because I commit to the result of the 
coin toss before I know my friend’s choice. Commitment is a very useful 
primitive (used in many places in cryptography) that enables a party 
to convincingly commit to a choice without revealing it until they 
choose to reveal it. The party should not be able to change their 
mind after the commitment which is what the scheme guarantees. 
\end{solution}

\clearpage