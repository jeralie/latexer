\q{5}{Potpourri}

\begin{enumerate}
\item Can we define an algorithm to enumerate the elements of the hypercube?


\solution{
No: think about the bitstring representations of the vertices, they have one bit per dimension, which we can prove to be unenumerable by Cantor’s diagonalization argument.
}
% \vspace{4cm}

\item Find a bijection between two distinct sets of vertices of the hypercube.

\solution{
There are many solutions. A bijection uniquely pairs elements of the set. 
For example $f(0x) = 1x$, $f(1x) = 0x$ where $0x$ is the vertex with a 
bit string representation starting with 0 followed by a bitstring $x$.
}
% \vspace{4cm}

\item Using the bitstring labels for vertices: how many of the vertices have
 labels whose digits sum to 5?
%  \begin{solution}{3cm}
% \end{solution}
% \vspace{4cm}

\solution{
If there were finite dimensions, this would be $\choose{n}{5}$ 
(since we must pick 5 digits to be 1, the rest are 0). With infinite 
dimensions, there are an infinite number of vertices. It is countable 
since we can represent each choice as a subset of $\mathbb{R}^5$ 
(the indices of the 5 1s) (we exclude vectors with repeated elements).
}


\item Define a bijection between one of $\mathrm{N}$ or $\mathrm{R}$ to 
the \textit{vertices} of the hypercube.
% \solutionimage{

\solution{
We define $\mathrm{R} \to E$ as the binary representation 
of the elements of $\mathrm{R}$, which are the bitstring labels for $E$.
% }
% \vspace{4cm}
}

\item Define a bijection between one of $\mathrm{N}$ or $\mathrm{R}$ to 
the \textit{edges} of the hypercube.
% \solutionimage{

\solution{
We define the injection $\mathbb{R} \to \mathbb{R} \times 
\mathbb{N} \leftrightarrow \mathbb{R}$. The edge $(x, n) \in 
\mathbb{R} \times \mathbb{N}$. Represents the edge between the 
vertex $x$ and the vertex which is bit-flipped on dimension $n$. 
There are precisely 2 $(x, n)$ for each $e \in E$. Since the set is infinite, the two are bijective.
}
% }
\end{enumerate}