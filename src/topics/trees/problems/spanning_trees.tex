\question Show that the edges of a complete graph on $n$ vertices 
for even $n$ can be partitioned into $\frac{n}{2}$ edge disjoint 
spanning trees. \newline
\textit{Hint}: Recall that a complete graph is an undirected graph 
with an edge between every pair of vertices. The complete graph has 
$\frac{n * (n-1)}{2}$ edges. A spanning tree is a tree on all $n$ 
vertices -- so it has $n-1$ edges. So the complete graph has enough 
edges (for even $n$) to create exactly $\frac{n}{2}$ edge disjoint 
spanning trees (i.e. each edge participates in exactly one spanning 
tree). You have to show that this is always possible.

\begin{solution}[5in]
We proceed by induction.\newline
\textit{Base Case}: Consider a complete graph on 2 vertices. This 
can clearly be partitioned into 2/2 = 1 edge disjoint spanning tree, 
because the graph is already a tree.\newline
\textit{Inductive Hypothesis}: Assume that the edges of a complete 
graph on $k$ vertices (for $k$ even) can be partitioned into 
$\frac{k}{2}$ edge disjoint spanning trees.\newline
\textit{Inductive Step}: We need to partition the edges of a 
complete graph $G_{k+2}$ on $k+2$ vertices into $\frac{k}{2} +1$ 
edge disjoint spanning trees.\newline
To do this, label the vertices of $G_{k+2}$ as $v_1, v_2, \dotsc ,v_{k+2}$. 
Remove the vertices $v_k+1$ and $v_{k+2}$ (and associated edges) to 
form a complete graph $G_k$ with $k$ vertices $v_1, \dotsc, v_k$. 
By the inductive hypothesis, $G_k$ has $\frac{k}{2}$ edge disjoint 
spanning trees; call these trees $T_1, \dotsc ,T_k=2$.
Add the vertices $v_{k+1}$ and $v_{k+2}$ back into $G_k$ to once 
again form the graph $G_{k+2}$. These vertices come with $2k+1$ 
extra edges, connecting $(v_i,v_{k+1})$ and $(v_i,v_{k+2})$ for 
each $i = 1,2,\dotsc,k,$ and also
$(v_{k+1}, v_{k+2})$. These edges must be included into spanning trees.
We wish to extend the trees $T_1,\dotsc,T_{k+2}$ to include the new 
vertices $v_{k+1}$ and $v_{k+2}$. To do this, for each tree $T_i$, 
attach two new edges $(v_i,v_{k+1})$ and $(v_{i+\frac{k}{2}}, v_{k+2})$. 
This extends each tree $T_i$ to be a spanning tree.
The remaining edges form one additional spanning tree. These edges 
are $(v_{i+\frac{k}{2}}, v_{k+1})$ and $(v_i,v_{k+2})$ for $i =1 to 
\frac{k}{2}$, along with the connecting edge $(v_{k+1}, v_{k+2})$. 
These edges connect each of the vertices $v_{k+1}$ and $v_{k+2}$ to 
half the remaining vertices, and together with the edge between 
$v_{k+1}$ and $v_{k+2}$ this gives the desired spanning tree.
Therefore, we have covered the graph in $\frac{k}{2} +1$ edge disjoint 
spanning trees. This completes the induction.\newline

Remark: The key idea here is the following:\newline
Take a graph with $k$ vertices that is partitioned into $\frac{k}{2}$ 
spanning trees. In the inductive step, we want to add two vertices 
(with associated edges). To maintain a partitioning into spanning trees, 
we must expand the preexisting $\frac{k}{2}$ trees to the new vertices, 
but this is a bit subtle!
We need to add the two new vertices to each of the preexisting 
$\frac{k}{2}$ trees, which takes 2  $\frac{k}{2}= k$ edges connecting 
the preexisting $k$ vertices to the two new vertices. It’s really 
important that we use only one edge out of each of the original vertices! 
This is because otherwise, we would use up both new edges out of one 
of the vertices $v_j$, but then our final new spanning tree wouldn’t 
be able to reach $v_j$, so the remaining $k+1$ edges wouldn’t be able 
to form a spanning tree!\newline
So to do this, we need to split the original $k$ vertices into two 
equal subsets of  $\frac{k}{2}$ vertices each, and connect each half 
to one of the two new vertices. Once we do that, we can then justify 
forming a new spanning tree from the remaining edges, which allows us 
to complete the argument.	
\end{solution}