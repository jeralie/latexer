\question We saw in the notes on page 8 that 1 and 2 above were saying 
the same thing- that is, stated rigorously, 1 $\Leftrightarrow$ 2. 
We will now prove that 1 $\Leftrightarrow$ 3:
\begin{solution}
First we prove the forward direction (1 $\rightarrow$ 3). If $G$ is 
connected and has no cycles, obviously the first part of 3 (connectedness) 
is satisfied. We now also prove that, in this case, the removal of any 
edge from $G$ must also disconnect it. Consider the possibility that 
removing an edge from $G$ did not disconnect it (point out that you 
are doing a proof by contradiction!). Consider the nodes on the 
endpoints of the removed edge, denoted $u$ and $v$. If there is still 
some path $P$ from $u$ to $v$ in $G$, then that path, when augmented 
with the removed edge, would form a cycle since we could start at $u$, 
use $P$ to arrive at $v$, and finally use the removed edge to arrive 
back at $u$ (from $v$). But this is a cycle, which is a contradiction! 
Hence, we are done with the forward direction. \newline
Now we prove the reverse direction: that is, that 3 $\rightarrow$ 1. 
We are now  given that $G$ is connected and that removing any edge 
will disconnect it. We now wish to prove that this implies 1. Obviously 
the first part of 1 (connectedness) is satisfied. Consider the 
possibility that $G$ did contain a cycle (again, point out that you 
are doing a proof by contradiction!). If this were the case, we could 
remove an edge in that cycle, and $G$ would still be connected. Why? 
Consider any node on the cycle. If we remove one of the edges from the 
cycle, then it means that the other nodes on that cycle are still 
reachable from one another (try showing this pictorially). But this 
contradicts the given that removing any edge disconnects $G$! Now we 
have proven the reverse direction, and we are done. 
\end{solution}

\clearpage