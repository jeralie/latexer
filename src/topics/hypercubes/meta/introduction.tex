\item \textbf{Introduction}
\begin{enumerate}[label=*]
\item The first definition is a recursive one, the second one is the labeling of vertices with binary numbers
\item The recursive definition is usually easier to see, so start with that one. Draw a single vertex, pointing out that this is a 0-dimensional hypercube. Draw a second 0-dimensional hypercube and connect them. Now you have a 1-dimensional hypercube. Draw a second 1-dimensional hypercube, match up the vertices and you have a 2-dimensional hypercube, etc. Have students right out the general definition after seeing some of these--that a n-dimensional hypercube is made up of two n-1-dimensional hypercubes that are connected at each vertex.
Next, bring up the binary definition of hypercubes. A good way to introduce this would be as follows: draw a 1-dimensional hypercube, label the two vertices 0 and 1. Now draw a second 1-dimensional hypercube labeled the same way. Now, when you connect the two hypercubes, take all the labels on 1 of them, and put a 0 in front of each one, and put a 1 in front of each label on the second one. This results in a square with vertices spanning from 00 to 11. This is a good time to show that to get from one vertex to a neighbor, you only need to flip one bit. You can also point out that this kind of looks like cartesian coordinates, with 00 being next to 01 and 10, and so forth. Maybe extend this two 3-dimensional hypercube to be sure everyone follows the process, adnthat should be all you’ll need to do for definitions.
Mention that knowing both definitions is very useful since they each have advantages when doing proofs. In general, the recursive definition works really well with inductive proofs, while the binary definition works better with direct proofs. 
\end{enumerate}