\noindent\fbox{\begin{minipage}{\dimexpr\textwidth-2\fboxsep-2\fboxrule\relax}
\vspace{3 mm}
 What is an $n$ dimensional hypercube? 
\begin{enumerate}[label= ]
\item \fontsize{12pt}{30pt}\selectfont
\textbf{Bit definition}: Two  \line(1,0){70}  $x$ and $y$ are \line(1,0){70} 
and only if \line(1,0){30} and \line(1,0){30} differ in \line(1,0){150} bit 
position. 
\item \textbf{Recursive definition}: Define the 0-\line(1,0){70} as the $(n-1)$ 
dimensional \line(1,0){70} \newline with vertices labeled 0x (x is an element 
of \line(1,0){100} (hint: how many remaining bits are there?). Do the same 
for the 1-\line(1,0){70} with vertices labeled \line(1,0){50}. Then an 
$n$ dimensional \line(1,0){70} is created by placing an edge between 
\line(1,0){50} and \line(1,0){50} in the \line(1,0){90}  and 
\line(1,0){90} \newline respectively.
\end{enumerate}
\end{minipage}}

\begin{solution}
\textbf{Bit definition}: vertices, adjacent, $x$, $y$, exactly one


\textbf{Recursive definition}: subcube, hypercube, $(0, 1)^{n-1}$, 
subcube, 1x, hypercube, 0x, 1x, 0-subcube, 1-subcube
\end{solution}