\question Prove that any cycle in an $n$-dimensional hypercube must 
have even length.
\begin{solution}[3in]
Answer: Here are three ways to solve this problem: here we will argue 
via bit flips, but there also exist arguments using the parity of 
Hamming distance, or induction on $n$. Note that induction on $n$ 
is more difficult and prone to build-up error.
					
Answer 1: Bit flips
					
Main idea: moving through an edge in a hypercube flips exactly one 
bit, and moreover each bit must be flipped an even number of times 
to end up at the starting vertex of the cycle.
					
Proof: Each edge of the hypercube flips exactly one bit position. 
Let $E_i$ be the set of edges in the cycle that flip bit $i$. Then 
$|E_i|$ must be even. This is because bit $i$ must be restored to 
its original value as we traverse the cycle, which means that bit 
$i$ must be flipped an even number of times. Since each edge of the 
cycle must be in exactly one set $E_j$, the total number of edges in 
the cycle = $\sum_j |E_j|$ is a sum of even numbers and therefore even. 
				
\end{solution}
