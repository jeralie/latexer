\question An officer stored an important letter in her safe. In case 
she is killed in battle, she decides to share the password with her 
troops. Everyone knows there are 3 spies among the troops, but no one 
knows who they are except for the three spies themselves.The 3 spies 
can coordinate with each other and they will either lie and make people 
not able to open the safe, or will open the safe themselves if they can. 
Therefore, the officer would like a scheme to share the password that 
satisfies the following conditions: 
 \begin{enumerate}
\item When $M$ of them get together, they are guaranteed to be able to 
open the safe even if they have spies among them. 
\item The 3 spies must not be able to open the safe all by themselves. 
 \end{enumerate}
 
Please help the officer to design a scheme to share her password. What 
is the scheme? What is the smallest $M$? Show your work and argue why 
your scheme works and any smaller $M$ couldn’t work.

\begin{solution}
The key insight is to realize that both polynomial-based secret-sharing 
and polynomial-based error correction work on the basis of evaluating 
an underlying polynomial at many points and then trying to recover that 
polynomial. Hence they can be easily combined. Suppose the password is 
$s$. The officer can construct a polynomial $P(x)$ such that $s = P(0)$ and 
share $(i,P(i))$ to the $i$-th person in her troops. Then the problem is: 
what should the degree of $P(x)$ be and what is the smallest $M$? 
First, the degree of polynomial $d$ should not be less than 3. It is 
because when $d < 3$, the 3 spies can decide the polynomial $P(x)$ uniquely. 
Thus, $n$ will be at least 4 symbols. Let’s choose a polynomial $P(x)$ of 
degree 3 such that $s = P(0)$. We now view the 3 spies as 3 general errors. 
Then the smallest $M = 10$ since n is at least 4 symbols and we have $k = 3 $
general errors, leading us to a “codeword” of $4+2 \cdot 3 = 10$ symbols 
(or people in our case). Even though the 3 spies are among the 10 people 
and try to lie on their numbers, the 10 people can still be able to 
correct the $k = 3$ general errors by the Berlekamp-Welch algorithm and 
find the correct $P(x)$. 

Alternative solution: Another valid approach is making $P(x)$ of degree 
$M − 1$ and adding 6 public points to deal with 3 general errors from 
the spies. In other words, in addition to their own point $(i,P(i)), $
 everyone also knows the values of 6 more points, 
$(t +1,P(t +1)),(t +2,P(t +2)), \dotsc ,(t +6,P(t +6))$, where $t$ is the number 
of the troops. The spies have access to total of $3+6 = 9$ points so the 
degree $M - 1 $ must be at least 9 to prevent the spies from opening the 
safe by themselves. Therefore, the minimum $M$ is 10.
\end{solution}