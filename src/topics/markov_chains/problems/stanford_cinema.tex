\question \textbf{Stanford Cinema} \\*
You have a database of an infinite number of movies. Each movie has a 
rating that is uniformly distributed in {0, 0.5, 1, 1.5, 2, 2.5, 3, 
3.5, 4, 4.5, 5} independent of all other movies. You want to find two 
movies such that the sum of their ratings is greater than 7.5 (7.5 is 
not included).
\begin{enumerate}[label=\alph*)]
\item
A Stanford student chooses two movies each time and calculates the sum 
of their ratings. If is less than or equal to 7.5, the student throws 
away these two movies and chooses two other movies. The student stops 
when he/she finds two movies such that the sum of their ratings is 
greater than 7.5. What is the expected number of movies that this 
student needs to choose from the database?

\begin{solution}[4cm]
Each time when the Stanford student chooses two movies, there are $11^2 
= 121$ different possible pairs of ratings. By simple counting, we 
know that there are 15 pairs whose sum is greater than $7.5$. Therefore, 
the probability that in a single trial, the Stanford student gets two 
movies such that the sum of their ratings is greater than $7.5$ is 
$\frac{15}{121}$. Then the number of times that the student needs to 
pick movies is geometrically distributed with mean $\frac{121}{15}$. 
Then the expected number of movies that the student needs to choose 
is $\frac{242}{15} \approx 16.13$.
\end{solution}

\item
A Berkeley student chooses movies from the database one by one and 
keeps the movie with the highest rating. The student stops when he/she 
finds the sum of the ratings of the last movie that he/she has chosen 
and the movie with the highest rating among all the previous movies is 
greater than 7.5. What is the expected number of movies that the 
student will have to choose?
\begin{solution}[9cm]
We use a Markov chain to represent to process that the Berkeley student 
gets two desired movies. There are 11 possible ratings:
\begin{equation*}
S = \{0, 0.5, 1, 1.5, 2, 2.5, 3, 3.5, 4, 4.5, 5 \}
\end{equation*}
We divide the set of ratings into 2 subsets, $L = \{0, 0.5, 1, 1.5, 2, 
2.5\}$ and $H = \{3, 3.5, 4, 4.5, 5\}$. Since the goal of the student 
is to get two movies that the sum of their rating is greater than 7.5, 
the movies whose ratings are in the set $L$ have no contribution to 
this goal. Then we can use 7 states $\{s_L, s_3, s_{3.5}, s_4, s_{4.5}, 
s_5, s_E\}$ to represent the progress to get two movies such that the 
sum of their ratings is greater than 7.5. The state $s_L$ denotes the 
cases when highest movie rating that the Berkeley student has got is 
in $L$. The states $s_i, 3 \leq i \leq 5$, denote the cases when the 
highest movie rating that the student has got is $i$. The state $s_E$ 
denotes the case when the student has got two movies such the sum of 
their ratings is greater than 7.5 and the choosing process ends. We 
can see that the process is a Markov chain with probability transition 
matrix $P$ as follows: \newpage
$P = \frac{1}{11}
\begin{bmatrix}
\begin{array}{ccccccc|c}
6 & 1 & 1 & 1 & 1 & 1 & 0 & s_L \\
0 & 7 & 1 & 1 & 1 & 0 & 1 & s_3 \\
0 & 0 & 8 & 1 & 0 & 0 & 2 & s_{3.5} \\
0 & 0 & 0 & 8 & 0 & 0 & 3 & s_4 \\
0 & 0 & 0 & 0 & 7 & 0 & 4 & s_{4.5} \\
0 & 0 & 0 & 0 & 0 & 6 & 5 & s_5 \\
0 & 0 & 0 & 0 & 0 & 0 & 11 & s_E \\
\hline
s_L & s_3 & s_4 & s_{4.5} & s_5 & s_E
\end{array}
\end{bmatrix}$ \\
Let $F_s$ be the expected time to get to state $s_E$, starting from 
state $s$,
$F = 
\begin{bmatrix}
F_{s_L} & F_{s_3} & F_{s_{3.5}} & F_{s_4} & F_{s_{4.5}} & F_{s_5} \\
\end{bmatrix}^T$
$P'$ be the sub-matrix of $P$ consisting of the first 6 columns and rows 
of $P$ (which has 7), and $U$ be an all-one column vector with length 6. 
We have the first step equations 
\begin{equation*}
F = P'F + U
\end{equation*}
Solving these linear equations, we get \\
$F = 
\begin{bmatrix}
6.0164 & 5.5764 & 4.8889 & 3.666667 & 2.75 & 2.2 
\end{bmatrix}^T$ \\
The initial state is $s_L$ with probability $\frac{6}{11}$ and is 
$s_i, 3 \leq i \leq 5$ with probability $\frac{1}{11}$ respectively. 
Then we know that the expected number of movies that the student 
needs to choose is 
\begin{equation*}
1 + \frac{6}{11}F_{s_L} + \frac{1}{11}(F_{s_3} + F_{s_{3.5}} + F_{s_4} 
+ F_{s_{4.5}} + F_{s_5}) = 6.02
\end{equation*}
This shows that the Berkeley student is smarter than the Stanford student.
\end{solution}

\end{enumerate}