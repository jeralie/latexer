{\tabulinesep=1mm
\begin{tabu}{|p{16cm} |}
\hline
We want to send $n$ packets and we know that $k$ packets could get lost. \newline
\begin{center}
\includegraphics[width=8cm, height=0.7cm]{erasure_intro.jpg}
\end{center}

How many more points does Alice need to send to account for $k$ possible errors?
\line(1,0){10}
\begin{solution}
$k$. Since we have constructed an $n - 1$ degree polynomial we will need $n$ points to recover the polynomial, so $n - k + k = n$.
\end{solution}

What degree will the resulting polynomial be? \line(1,0){15}
\begin{solution}
$n - 1$. We need the polynomial to encode all of the degrees of freedom from the packets. A $k$ degree polynomial has $k + 1$ degrees of freedom, so for $n$ degrees of freedom we need a polynomial of degree $n - 1$.
\end{solution}

How large should $q$ be if Alice is sending n packets with k erasure errors, where each packet has $b$ bits?
\begin{solution}
Modulus should be larger than $n+k$ and larger than $2^b$ and be prime since these will be to domain and range of our polynomial function (the two axes on the graph representing it).
\end{solution}

What would happen if Alice instead send $n + k - 1$? Why will Bob be unable to recover the message?
\begin{solution}
Bob will receive $n - 1$ distinct points and needs to reconstruct a 
polynomial of degree $n - 1$. By Fact \#3 this is impossible. There are 
$q$ polynomials of at most degree $n - 1$ in $GF(q)$ that go through the 
$n - 1$ points that Alice sent. 
\end{solution}
\\
\hline
\end{tabu}
}
