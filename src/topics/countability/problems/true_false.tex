\question \textbf{True/False} 
\begin{enumerate}[label=(\alph*)]
\item Every infinite subset of a countable set is countable
\begin{solution}
True. Define $E$ as the subset. Define function $f$ where $f(1) = min(E)$, 
$f(k) = k$th smallest element of $E$. We see a bijective mapping clearly 
exists between $f$ and $E$. and since the $x$-values in $f$ are just the natural 
numbers, there is a mapping between $E$ and $N$.
\end{solution}

\item If $A$ and $B$ are both countable, then $A \times B$ is countable
\begin{solution}
True. Can draw a bijection where first elem of $A$ maps to 1, first elem 
of $B$ maps to 2, second elem of $A$ maps to 3, etc. This will include $A \times B$
at the end, and because there is a bijection from $A$ to $N$ and $B$ to $N$, there is a bijection here from $A \times B$ to $N$. There is clearly mapping from 
$A \times B$ to $N \times N$, and $N \times N$ to $N$, so there is a 
bijective mapping from $A \times B$ to $N$
\end{solution}

\item Every infinite set that contains an uncountable set is uncountable. 
\begin{solution}
True
Let $A$ be an uncountable subset of $B$.
Assume that $B$ is countable.
$f: N \rightarrow B$ is a bijection.
There must be a subset of $N$ such that $f: M \rightarrow A$ is a bijection
Then $A$ is countable.
This is a contradiction.
So $B$ must be uncountable.
\end{solution}
\end{enumerate}