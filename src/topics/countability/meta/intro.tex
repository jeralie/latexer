\item \textbf{Cantor-Bernstein Theorem}
\begin{itemize}
\item This theorem is not focused on (maybe not even really mentioned) 
in lecture, so the name and formal statement is just a neat thing to 
mention, BUT the technique that comes from it, namely the technique used 
to prove $Q$s countability, is very useful. Said technique states that 
if $|A| \leq |B|$ and $|B| \leq |A|$ then $|A| = |B|$.
\item You can do a proof of why $Q$ is countable here.
\item We know that $|N| \leq |Q|$ because every natural number is a rational number.
\item Just need to show that $|Q| \leq |N|$.
\item Feel free to do a short version of the spiral proof. They will 
probably go over this in lecture, but if they didn’t get it there you 
should cover it here. (Ask!)
\end{itemize}